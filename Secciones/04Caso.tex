%%%%%%%%%%%%%%%%%%%%%%%%%%%%%%%%%%%%%%%%%%%%%%%%%%%%%%%%
\fondo{celeste}
\section{Caso de uso}
\fondo{blanco}
%%%%%%%%%%%%%%%%%%%%%%%%%%%%%%%%%%%%%%%%%%%%%%%%%%%%%%%%

%%%%%%%%%%%%%%%%%%%%%%%%%%%%%%%%%%%%%%%%%%%%%%%%%%%%%%%%
\begin{frame}
    \frametitle{Caso de uso}

    \begin{itemize}
        \item \textbf{Asignatura:} Cálculo Diferencial e Integral
        \item \textbf{Carrera:} Ciencia de Datos
        \item \textbf{Nivel:} Segundo nivel
        \item \textbf{Trabajo:} Artículo titulado \textit{¿ChatGPT sabe Cálculo diferencial?}
        \item \textbf{Objetivo:} Evaluar las respuestas de ChatGPT sobre la historia y los procedimientos del cálculo diferencial.
    \end{itemize}
\end{frame}

%%%%%%%%%%%%%%%%%%%%%%%%%%%%%%%%%%%%%%%%%%%%%%%%%%%%%%%%

%%%%%%%%%%%%%%%%%%%%%%%%%%%%%%%%%%%%%%%%%%%%%%%%%%%%%%%%
\begin{frame}
    \frametitle{Diseño de la actividad}

    \small
    \begin{enumerate}[leftmargin=*, label=\arabic*.]
        \item Interrogar a ChatGPT sobre la historia del Cálculo desde dos cuentas distintas.
        \item Evaluar la veracidad de las respuestas con bibliografía académica.
        \item Solicitar a ChatGPT la resolución de ejercicios, incluyendo, entre otras:
        \begin{itemize}
            \item Derivada por definición
            \item Reglas de derivación
        \end{itemize}
        \item Verificar si las respuestas son correctas o contienen errores.
        \item Justificar cada error identificado y reflexionar sobre su origen.
        \item Presentar todo en un artículo estructurado, con citas y conclusiones.
    \end{enumerate}
\end{frame}

%%%%%%%%%%%%%%%%%%%%%%%%%%%%%%%%%%%%%%%%%%%%%%%%%%%%%%%%

%%%%%%%%%%%%%%%%%%%%%%%%%%%%%%%%%%%%%%%%%%%%%%%%%%%%%%%%
\begin{frame}
\centering
    \postitimg[0.6\linewidth]{Figuras/Fig07.png}
\end{frame}
%%%%%%%%%%%%%%%%%%%%%%%%%%%%%%%%%%%%%%%%%%%%%%%%%%%%%%%%

%%%%%%%%%%%%%%%%%%%%%%%%%%%%%%%%%%%%%%%%%%%%%%%%%%%%%%%%
\begin{frame}
\begin{columns}
\column{0.45\textwidth}
\begin{block}{}
    Los estudiantes «calificaron» las respuestas de ChatGPT.
\end{block}
\column{0.55\textwidth}
\centering
    \postitimg[0.95\linewidth]{Figuras/Fig08.png}
\end{columns}

\end{frame}
%%%%%%%%%%%%%%%%%%%%%%%%%%%%%%%%%%%%%%%%%%%%%%%%%%%%%%%%


%%%%%%%%%%%%%%%%%%%%%%%%%%%%%%%%%%%%%%%%
\begin{frame}{Hallazgos}
\begin{columns}
    \column{0.65\linewidth}
    \begin{itemize}[leftmargin=*]
        \item Los estudiantes demostraron alta \textbf{capacidad para identificar} y analizar errores.
        \item Detectaron \textbf{correlación} entre la \textbf{complejidad} de los ejercicios y la \textbf{precisión} de ChatGPT.
        \item Se fomentó el \textbf{pensamiento crítico} y la comprensión profunda de los conceptos matemáticos.
    \end{itemize}
    \column{0.35\linewidth}
    \postitimg[0.98\linewidth]{Figuras/Fig09.png}
\end{columns}
\end{frame}
