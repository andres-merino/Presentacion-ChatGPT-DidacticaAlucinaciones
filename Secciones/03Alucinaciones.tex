%%%%%%%%%%%%%%%%%%%%%%%%%%%%%%%%%%%%%%%%%%%%%%%%%%%%%%%%
\fondo{celeste}
\section{¿Qué son las alucinaciones en IA?}
\fondo{blanco}
%%%%%%%%%%%%%%%%%%%%%%%%%%%%%%%%%%%%%%%%%%%%%%%%%%%%%%%%

\begin{frame}[t]{¿Qué es una alucinación en IA?}

\vspace{-9mm}
\begin{columns}
\column{0.5\textwidth}
\begin{block}{Definición}
    En inteligencia artificial, una \textbf{alucinación} es una respuesta generada por un modelo que es \textbf{falsa o incorrecta}, pero expresada con gran seguridad, y que \textbf{no se justifica en los datos de entrenamiento}.
\end{block}
\column{0.5\textwidth}
\postitimg[0.9\linewidth]{Figuras/Fig06.png} 
\end{columns}
\end{frame}



\begin{frame}
\begin{block}{Ejemplos comunes}
\begin{itemize}
    \item \textbf{Errores fácticos}: datos inventados o fechas incorrectas.
    \item \textbf{Errores conceptuales}: definiciones mal formuladas.
    \item \textbf{Errores matemáticos}: pasos equivocados en cálculos o demostraciones.
    \item \textbf{Invención de fuentes}: citas o autores que no existen.
\end{itemize}
\end{block}

\vspace{0.3cm}
\pause
\begin{block}{Riesgo}
El lenguaje fluido puede ocultar el error y generar una falsa sensación de autoridad.
\end{block}
\end{frame}

\begin{frame}
    \frametitle{¿Qué tan común son las alucinaciones?}
    \centering
    \postitimg[0.6\linewidth]{Figuras/Fig11.png}
\end{frame}


\begin{frame}[t]{Potencial didáctico de las alucinaciones}
\begin{block}{¿Por qué usarlas en el aula?}
\begin{itemize}
    \item Fomentan el \textbf{pensamiento crítico} y la actitud de verificación.
    \item Permiten ejercicios de \textbf{análisis y depuración de errores}.
    \item Estimulan la \textbf{discusión argumentada} sobre conceptos.
    \item Refuerzan la comprensión al contrastar respuestas correctas e incorrectas.
\end{itemize}
\end{block}

\vspace{0.3cm}
\pause
\begin{block}{En resumen}
Una alucinación bien dirigida puede convertirse en una herramienta de aprendizaje profundo.
\end{block}
\end{frame}
